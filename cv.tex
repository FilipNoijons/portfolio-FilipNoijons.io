% Options for packages loaded elsewhere
\PassOptionsToPackage{unicode}{hyperref}
\PassOptionsToPackage{hyphens}{url}
%
\documentclass[
]{article}
\usepackage{amsmath,amssymb}
\usepackage{iftex}
\ifPDFTeX
  \usepackage[T1]{fontenc}
  \usepackage[utf8]{inputenc}
  \usepackage{textcomp} % provide euro and other symbols
\else % if luatex or xetex
  \usepackage{unicode-math} % this also loads fontspec
  \defaultfontfeatures{Scale=MatchLowercase}
  \defaultfontfeatures[\rmfamily]{Ligatures=TeX,Scale=1}
\fi
\usepackage{lmodern}
\ifPDFTeX\else
  % xetex/luatex font selection
\fi
% Use upquote if available, for straight quotes in verbatim environments
\IfFileExists{upquote.sty}{\usepackage{upquote}}{}
\IfFileExists{microtype.sty}{% use microtype if available
  \usepackage[]{microtype}
  \UseMicrotypeSet[protrusion]{basicmath} % disable protrusion for tt fonts
}{}
\makeatletter
\@ifundefined{KOMAClassName}{% if non-KOMA class
  \IfFileExists{parskip.sty}{%
    \usepackage{parskip}
  }{% else
    \setlength{\parindent}{0pt}
    \setlength{\parskip}{6pt plus 2pt minus 1pt}}
}{% if KOMA class
  \KOMAoptions{parskip=half}}
\makeatother
\usepackage{xcolor}
\usepackage[margin=1in]{geometry}
\usepackage{graphicx}
\makeatletter
\def\maxwidth{\ifdim\Gin@nat@width>\linewidth\linewidth\else\Gin@nat@width\fi}
\def\maxheight{\ifdim\Gin@nat@height>\textheight\textheight\else\Gin@nat@height\fi}
\makeatother
% Scale images if necessary, so that they will not overflow the page
% margins by default, and it is still possible to overwrite the defaults
% using explicit options in \includegraphics[width, height, ...]{}
\setkeys{Gin}{width=\maxwidth,height=\maxheight,keepaspectratio}
% Set default figure placement to htbp
\makeatletter
\def\fps@figure{htbp}
\makeatother
\setlength{\emergencystretch}{3em} % prevent overfull lines
\providecommand{\tightlist}{%
  \setlength{\itemsep}{0pt}\setlength{\parskip}{0pt}}
\setcounter{secnumdepth}{-\maxdimen} % remove section numbering
\ifLuaTeX
  \usepackage{selnolig}  % disable illegal ligatures
\fi
\IfFileExists{bookmark.sty}{\usepackage{bookmark}}{\usepackage{hyperref}}
\IfFileExists{xurl.sty}{\usepackage{xurl}}{} % add URL line breaks if available
\urlstyle{same}
\hypersetup{
  pdftitle={Curriculum Vitae},
  hidelinks,
  pdfcreator={LaTeX via pandoc}}

\title{Curriculum Vitae}
\author{}
\date{\vspace{-2.5em}}

\begin{document}
\maketitle

\hypertarget{persoonlijke-gegevens}{%
\section{Persoonlijke gegevens}\label{persoonlijke-gegevens}}

\begin{itemize}
\tightlist
\item
  \textbf{Naam:} Filip Bastiaan Noijons
\item
  \textbf{Adres:} Fruinplantsoen 98, 3571PV
\item
  \textbf{Telefoon:} 06 81940552
\item
  \textbf{E-mail:}
  \href{mailto:FilipNoyons@gmail.com}{\nolinkurl{FilipNoyons@gmail.com}}
\item
  \textbf{LinkedIn:}
  \url{https://www.linkedin.com/in/filip-noijons-0073881b4/}
\item
  \textbf{Geboortedatum:} 8 augustus 2000
\end{itemize}

\hypertarget{profielschets}{%
\section{Profielschets}\label{profielschets}}

Een beknopte samenvatting van je vaardigheden, ervaring en doelen.

\hypertarget{opleidingen}{%
\section{Opleidingen}\label{opleidingen}}

\hypertarget{bachelor-in-life-sciences}{%
\subsection{Bachelor in Life Sciences}\label{bachelor-in-life-sciences}}

\begin{itemize}
\tightlist
\item
  \textbf{Hogeschool Utrecht:} HU, Utrecht
\item
  \textbf{Jaar:} 2020 - 2025
\end{itemize}

\hypertarget{master-in-naam-van-de-studie}{%
\subsection{Master in {[}Naam van de
Studie{]}}\label{master-in-naam-van-de-studie}}

\begin{itemize}
\tightlist
\item
  \textbf{Universiteit:} Universiteit Y, Plaats
\item
  \textbf{Jaar:} 20XX - 20XX
\end{itemize}

\hypertarget{werkervaring}{%
\section{Werkervaring}\label{werkervaring}}

\hypertarget{functie-functietitel}{%
\subsection{Functie: {[}Functietitel{]}}\label{functie-functietitel}}

\begin{itemize}
\tightlist
\item
  \textbf{Bedrijf:} Bedrijfsnaam, Plaats
\item
  \textbf{Periode:} Maand Jaar - Maand Jaar
\item
  \textbf{Taken:}

  \begin{itemize}
  \tightlist
  \item
    Beschrijf hier je verantwoordelijkheden en prestaties.
  \end{itemize}
\end{itemize}

\hypertarget{functie-functietitel-1}{%
\subsection{Functie: {[}Functietitel{]}}\label{functie-functietitel-1}}

\begin{itemize}
\tightlist
\item
  \textbf{Bedrijf:} Bedrijfsnaam, Plaats
\item
  \textbf{Periode:} Maand Jaar - Maand Jaar
\item
  \textbf{Taken:}

  \begin{itemize}
  \tightlist
  \item
    Beschrijf hier je verantwoordelijkheden en prestaties.
  \end{itemize}
\end{itemize}

\hypertarget{vaardigheden}{%
\section{Vaardigheden}\label{vaardigheden}}

\begin{itemize}
\tightlist
\item
  \textbf{Programmeertalen:} bijv. R, Python, etc.
\item
  \textbf{Software:} bijv. Microsoft Office, Adobe Suite, etc.
\item
  \textbf{Talen:} bijv. Nederlands (moedertaal), Engels (vloeiend), etc.
\end{itemize}

\hypertarget{projecten}{%
\section{Projecten}\label{projecten}}

\begin{itemize}
\tightlist
\item
  \textbf{Projectnaam:} Korte beschrijving van het project en je
  bijdrage.
\item
  \textbf{Periode:} Maand Jaar - Maand Jaar
\item
  \textbf{Technologieën/tools gebruikt:} bijv. Github, SQL, etc.
\end{itemize}

\hypertarget{certificeringen}{%
\section{Certificeringen}\label{certificeringen}}

\begin{itemize}
\tightlist
\item
  **Naam van
\end{itemize}

\end{document}
